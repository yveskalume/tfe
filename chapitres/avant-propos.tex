\chapter*{AVANT-PROPOS}
\justifying
\large
\setlength{\parindent}{2.5em}

Dans un monde en constante évolution, la sécurité personnelle reste une préoccupation cruciale pour les individus. L'essor de la technologie a ouvert de nouvelles perspectives pour répondre à cette préoccupation, offrant la possibilité de concevoir des systèmes d'appel au secours innovants et efficaces.\\

L'enlèvement est un crime grave qui suscite une inquiétude croissante dans de nombreuses régions du monde, y compris Lubumbashi, une ville en constante expansion. Face à cette réalité préoccupante, il est impératif de développer des solutions technologiques innovantes pour renforcer la sécurité personnelle et la réactivité des forces de l'ordre.\\


Ce travail de recherche abordera les étapes clés de la conception d'un tel système, en commençant par une analyse critique des solutions existantes, en passant par la proposition de notre propre solution, et en concluant par sa mise en œuvre.\\

Notre objectif ultime est de contribuer à la sécurité des citoyens en offrant un outil puissant et accessible qui pourrait potentiellement sauver des vies. Nous sommes conscients des défis techniques et des enjeux sociaux associés à un tel projet, mais c'est avec une conviction profonde que nous entreprenons cette mission. L'innovation technologique peut être un puissant catalyseur de changement, et nous sommes déterminés à explorer comment elle peut être mise au service de la sécurité des individus.\\

Au fil des prochaines pages, nous vous invitons à plonger dans notre exploration du développement d'une solution répondant à un besoin criant dans notre société contemporaine : la sécurité personnelle. Bien que cette étude soit circonscrite dans le temps à l'année académique actuelle, notre ambition dépasse ces limites. Nous aspirons à contribuer à la création d'un environnement plus sûr pour nos concitoyens, en proposant une solution qui, nous l'espérons, pourra servir de modèle et inspirer d'autres initiatives de sécurité dans d'autres contextes.

