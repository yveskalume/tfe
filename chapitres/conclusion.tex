\chapter*{CONCLUSION GÉNÉRALE}
\addcontentsline{toc}{chapter}{CONCLUSION GÉNÉRALE}
\justifying
Nous arrivons au terme de notre travail consacré à la conception d'un système d’appel au secours en se basant sur le cas d'enlèvement dans la ville de Lubumbashi. Nous avons parcouru un chemin jalonné de défis, d'analyses approfondies et de développements techniques pour répondre à la problématique de renforcer la sécurité et le bien-être de la communauté.\\

Nous avons avancé l'hypothèse qu'un système d'appel au secours, permettant à une victime d'enlèvement de déclencher discrètement un appel à l'aide, pourrait jouer un rôle clé dans l'amélioration de la sécurité de la ville. Pour atteindre cet objectif, nous avons exploré et développé des solutions bases sur la commande vocale et la reconnaissance sonore permettant d'alerter efficacement les contacts d'urgence et envoyer sa localisation en cas de danger imminent.\\

En fin de compte, nous espérons que notre travail contribuera à améliorer la sécurité dans la ville de Lubumbashi et à offrir aux résidents une nouvelle couche de protection en cas de danger. Cependant, nous sommes conscients que toute œuvre est teintée d'imperfections, et nous encourageons d'autres chercheurs à explorer ce même chemin pour éclaircir davantage les zones d'ombre qui subsistent.
Notre travail ne s'achève pas ici, il marque le début d'un engagement continu envers la sécurité et le bien-être de la communauté.
