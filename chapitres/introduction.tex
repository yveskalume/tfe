\chapter*{INTRODUCTION GÉNÉRALE}
\addcontentsline{toc}{chapter}{INTRODUCTION GÉNÉRALE}
\justifying
\setlength{\parindent}{2.5em}

L'importance de la sécurité est un enjeu majeur dans notre société contemporaine. Les défis sécuritaires sont nombreux et complexes, des problèmes tels que les enlèvements, la violence urbaine et la criminalité organisée continuent de menacer la tranquillité et la stabilité des communautés à travers le monde.\\

Face à ces défis, il devient essentiel de rechercher et de mettre en œuvre des solutions innovantes pour garantir la sécurité et protéger les individus contre ces menaces. De nouvelles technologies, des approches préventives et des méthodes d'intervention efficaces peuvent jouer un rôle crucial dans la création d'un environnement plus sûr.
Dans le cadre de notre travail de fin d’études en Génie Logiciel, nous nous penchons sur la conception d’une solution qui contribuera à la lutte contre ce fléau en permettant à une victime d'enlèvement d'alerter ses proches et les services de sécurité, tout en laissant des traces qui faciliteront sa localisation.

\section{Problématique}

La ville de Lubumbashi fait face à une situation sécuritaire de plus en plus préoccupante, marquée par une série d'enlèvements. En février 2023, un article publié par 7sur7.cd a révélé la découverte de huit cadavres dans divers quartiers de la ville en l'espace d'une semaine\footnote{https://7sur7.cd/2023/02/13/lubumbashi-8-corps-sans-vie-ramasses-en-une-semaine}. Un autre article de Radio Okapi publié 19 mai de cette même année met également en évidence la résurgence de la criminalité dans la ville de Lubumbashi depuis le début de l'année\footnote{https://www.radiookapi.net/2023/05/19/actualite/securite/lubumbashi-la-commission-justice-et-paix-de-leglise-catholique-denonce }. La Commission Justice et Paix de l'archidiocèse de Lubumbashi a dénoncé publiquement cette situation préoccupante, citant des cas de corps sans vie retrouvés et l'enlèvement récent d'une religieuse.  Ces cas ne représentent malheureusement qu'un exemple parmi tant d'autres.\\

Les enlèvements continuent de représenter une menace constante pour la sécurité des habitants de Lubumbashi, laissant place à de sérieuses inquiétudes concernant la stabilité et la quiétude de la ville.
Cette situation soulève des questions cruciales qui nécessitent une attention approfondie dans le cadre de notre travail

\begin{itemize}
	\item Comment peut-on développer des mécanismes discrets et efficaces pour permettre à une victime d'enlèvement d'alerter ses proches et les services de sécurité sans éveiller les soupçons du ravisseur ? 
	
	\item Comment peut-on aider les victimes d'enlèvement à laisser des traces ou des indices qui pourraient aider les forces de sécurité à les localiser ?
	
\end{itemize}

\section{Hypothèse}
Face au problème soulevé, nous pouvons répondre aux questions posées avec les solutions ci-après :

\begin{itemize}
\item Nous proposons de développer une application mobile permettant à une victime d'enlèvement de déclencher discrètement un appel à l'aide

\item Notre système contiendra également un système de géolocalisation pour permettre de localiser la victime.

\end{itemize}

\section{Choix et Intérêt du sujet}

\subsection{Choix du sujet}

Pour être formé à la science, il fallait produire un travail scientifique vérifié, accepté et justifié dans notre domaine de formation, un travail mis à la disposition de tous pour faire évoluer le domaine de notre science. Notre choix s'est porté sur la conception d'un système d’appel au secours grâce à la reconnaissance vocale appliquée aux cas d'enlèvement afin de permettre aux personnes qui sont victimes de kidnapping de lancer un appel à l'aide afin d'être secourues.

\subsection{Intérêt du sujet}

	\subsubsection{Intérêt du personnel}
	
	Nous avons choisi un cas concret afin de matérialiser les connaissances acquises tout au long de notre parcours académique. Mais, également mettre à la disposition des personnes en danger un outil qui leur permettra d’alerter les proches et la sécurité pour ainsi être secourues
	
	\subsubsection{Intérêt scientifique}
	
	Nous avons choisi ce sujet dans le but de satisfaire aux exigences académiques imposées à tous les étudiants qui doivent réaliser un travail de fin d'études. Cependant, l'obtention du diplôme n'est pas le seul objectif de ce travail, car nous souhaitons également qu'il serve de modèle fiable et utile pour les chercheurs qui viendront après nous.

\section{Méthodes et techniques}
\subsection{Méthodes}

Pour concrétiser ce travail, nous avons décidé d'utiliser le Dynamic Systems Development Method (DSDM). Le DSDM est une méthode agile qui met l'accent sur la livraison rapide de fonctionnalités et la collaboration étroite avec les parties prenantes.

Le DSDM se compose de plusieurs phases, dont voici les principales :

\begin{itemize}
	\item Étude de faisabilité : Cette phase consiste à établir les bases du projet, à définir son périmètre et à évaluer sa faisabilité. Elle permet de s'assurer que le projet est viable avant de s'engager pleinement.
	
	\item Étude business (ou analyse fonctionnelle) : Cette étape sert à la définition des spécifications, les fonctionnalités principales sont identifiées et priorisées en fonction de leur valeur ajoutée pour les utilisateurs.
	\item Modèle fonctionnel itératif : Dans cette phase, les fonctionnalités du système sont développées de manière itérative. Des courtes itérations permettent de tester rapidement et de recevoir les retours des parties prenantes, ce qui permet d'ajuster et d'améliorer le système au fil du temps.
	
	\item Mise en œuvre : Cette phase consiste à finaliser et à déployer le système dans son environnement de production.
\end{itemize}

\subsection{Techniques}
Pour pouvoir comprendre le fonctionnement du système que nous souhaitons implémenter, nous allons recourir à ces deux techniques qui sont :

\begin{itemize}
	\item La technique documentaire
	
	Nous allons utiliser la technique documentaire pour approfondir nos connaissances sur les concepts fondamentaux de la sécurité, la prévention des enlèvements, ainsi que les protocoles existant pour prévenir ce problème. Nous avons consulté des études académiques, des rapports gouvernementaux, des manuels de sécurité et d'autres ressources pertinentes pour nous aider à concevoir un système solide et efficace.
	
	\item La technique d’interview
	
	Nous opterons pour la technique d'interview afin de comprendre pleinement les problèmes et les défis auxquels les individus sont confrontés en matière de sécurité contre l'enlèvement.  Nous mènerons des entretiens avec des experts en sécurité, des responsables des forces de l'ordre, des professionnels spécialisés dans la protection des personnes et des individus ayant été victimes d'enlèvement. Grâce à ces entretiens, nous pourrons recueillir des informations essentielles pour concevoir un système d'alerte efficace qui répondra aux préoccupations et aux besoins réels des personnes en matière de sécurité.
\end{itemize}

\subsection{État de l'art}
Au cours des dernières années, plusieurs travaux ont été entrepris afin de lutter contre les problèmes d'insécurité liés au cas d'enlèvement : \\

Premièrement, nous avons consulté le travail intitulé “Implémentation D'un Système De Prévention Sur L'enlèvement En Taxi” défendu à l’École Supérieure d’Informatique Salama, durant l’année académique 2020-2021 par Kalonga Mwamba Jonathan, en vue de l’obtention d’un grade d’ingénieur technicien informatique dans la filière Génie Logiciel, Système Informatique. Dans son travail, l’auteur a proposé une solution constituée d’une application mobile qui permettra à l’utilisateur de scanner la plaque du véhicule dans lequel il veut monter, de lancer une alerte qui enverra des informations sur la localisation et le taxi a des personnes prédéfinies, ainsi qu'à un tableau de bord géré par un administrateur qui pourra voir sur une carte la géolocalisation du téléphone.\\

Le deuxième travail est “Application de la GSM dans la géolocalisation en cas de détresse” également défendu à l’École Supérieure d’Informatique Salama durant l’année académique 2016-2017 par Tela Matondo Nathan. Dans ce travail, l’auteur propose une solution qui permettra à une victime de signaler sa position à ses proches sans passer par internet et sans attirer l’attention de ses ravisseurs. Pour utiliser cette solution, l’utilisateur doit pré-enregistrer une liste des personnes, puis lorsqu’il monte dans un taxi, il suffira qu’il lance l’application mobile et cette dernière enverra périodiquement les coordonnées géographiques du smartphone aux contacts pré-enregistrés.\\

Enfin, une recherche intitulée “Voice-Controlled Tool for Anytime Safety of Women” \cite{voice_controlled} publiée par Agrawal Agrima, Maurya Ankita et Patil Amruta. Ce travail vise à aider les femmes qui se sentent vulnérables sur le lieu du travail en raison de l'environnement et de l'absence de moyens de se défendre en cas de situation difficile.\\

Pour surmonter ces problèmes, un bracelet intelligent à commande vocale a été conçu pour les femmes. Ce dispositif est activé par la voix de la victime et il commence à suivre la localisation de la victime en utilisant le module GPS de l'appareil, chaque fois que la commande vocale correspond, un message d'alerte est envoyé au numéro de contact pré-enregistré. Le GPS du téléphone mobile est utilisé pour localiser l'endroit le plus proche et envoyer un appel ou un SMS au poste de police le plus proche en mentionnant l'emplacement de la victime.

\subsection{Délimitation du travail}
Dans le souci et la quête d’un bon résultat, la restriction spatio-temporelle s’impose. Ainsi, nous limiterons notre champ de travail dans la province du Haut-Katanga et plus précisément dans la ville de Lubumbashi durant l’année académique 2022-2023.

\subsection{Subdivision du travail}
En plus de l'introduction générale et de la conclusion générale, notre travail sera divisé en 3 chapitres :

\begin{itemize}
	\item Chapitre 1 : Généralités et étude préalable : Ce chapitre traitera des concepts de base qui nous permettront de nous mettre d'accord sur les grandes lignes de notre travail.
	\item Chapitre 2 : Conception Et Modélisation: Cela évoquera des fondements importants qui nous permettront de voir de près les différentes facettes de la solution que nous essayons de proposer.
	\item Chapitre 3 : Processus Et Implémentation : Chapitre présentera les étapes pratiques à suivre pour arriver à notre solution.
\end{itemize}

\subsection{Outils logiciels et équipements utilisés}
\begin{itemize}
	\item Le langage de programmation Kotlin
	\item Le framework Angular, base sur Typescript
	\item Les IDEs IntelliJ IDEA et Android Studio
	\item Git et GitHub pour le versionnage de fichiers
	\item PlantUML et Draw.io
	\item Latex
\end{itemize}
